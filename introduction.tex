\section*{ВВЕДЕНИЕ}
\addcontentsline{toc}{section}{Введение}
Как захват людей, так и оценка позы человека на изображении имеют огромное количество приложений, как-то: наружное наблюдение, индексирование видео и т.д. Целью дипломного проекта является описание и реализация обобщенного фреймворка для захвата людей и оценки их позы на изображении, например, прохожих, но также и людей из спортивного видео.

Наилучшие на сегодняшний день методы для данных случаев не имеют ту же архитектуру, как метод приведенный в данном разделе. Благодаря тщательной разработке этого метода, он превосходит все современные методы на трех тестовых наборах даных, что используются в \cite{andriluka09}.

Этот фреймворк построен на основе моделей графических структур (pictorial structures), которая является мощной и обобщённой, и в то же время простой порождающей моделью, которая позволяет совершать точный и эффективный вывод соотношения частей. Также в фреймворке используются современные детекторы частей, которые позволяют достичь захвата в проблемных сценах.

В данном методе подсчитывается представление внешнего вида на основе дескриптора признаков Shape Context и используется AdaBoost для обучения дифференцирующих классификаторов частей.

\newpage
