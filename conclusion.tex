\section*{ЗАКЛЮЧЕНИЕ}
\addcontentsline{toc}{section}{Заключение}
В рамках данного дипломного проекта был реализован один из наиболее лучших на сегодняшний день методов захвата объектов со сложной конфигурацией на изображении и оценки их позы (положения), основанный на графических структурах.

В ходе разработки на практике было исследовано применение оператора обнаружения краев Канни, дескриптора признаков Shape Context, метода усиления (комбинирования) простых классификаторов AdaBoost, графических структур для моделирования объектов с сложной пространственной конфигурацией частей.

Разработанная система полностью удовлетворяет требованиям, сформуливанным в исходных данных к дипломному проекту, и обладает рядом положительных характеристик:
\begin{itemize}
  \item общность, то есть возможность применения данного программного приложения в широком спектре задач, что существенно отличает задействованный метод от других современных методов,
  \item относительная простота как реализации, так и математической формулировки,
  \item устойчивость к зашумленности изображений, благодаря предварительному небольшому размытия изображения,
  \item устойчивость детекторов частей объекта к различным аффинным преобразованиям, как-то: перенос, масштабирование, поворот и т.д.
\end{itemize}

Существуют возможности усовершенствования проекта с целью повышения точности и надежности поиска объектов на изображении. Одним из способов является использование цветовой информации для улучшения детекторов частей объектов.

Другим способом является дополнительная информация о связях частей объекта; мы же моделировали только кинематические ограничения. Это позволит, например, более устойчиво моделировать заграждение (заслонение) объекта каким-либо предметом или даже другим объектом.

В работе было проведено технико-экономическое обоснование разработки системы. Произведенные расчеты показали, что разработка программного обеспечения является рентабельной. Программное обеспечение имеет короткий срок окупаемости.

Разработанная система хорошо себя зарекомендовала, и по результатам тестирования был сделан вывод о возможности ее практического использования.

\newpage
