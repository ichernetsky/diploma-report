\thispagestyle{empty}

\begin{singlespace}
  {\small
    \begin{center}
      \begin{minipage}{0.8\textwidth}
        \begin{center}
          {\normalsize ОТЗЫВ}\\[0.2cm]
          на дипломный проект студента\\
          факультета компьютерных систем и сетей\\
          Учреждения образования ``Белорусский государственный университет информатики и радиоэлектроники''\\
          Чернецкого Ивана Николаевича\\
          на тему: ``Методы распознавания, захвата и сопровождения движущихся объектов и их применение в задаче отслеживания людей''
        \end{center}
      \end{minipage}
    \end{center}

    На время дипломного проектирования перед студентом Чернецким И.Н. была поставлена задача разработать программное обеспечение для захвата людей на изображении и оценки их позы. Тема является актуальной, так как количество применений захвата объектов на изображении, например, в наружном наблюдении или индексации видео с каждым днем растет.

    Чернецкий И.Н. на основании анализа большего количества специализированной литературы произвел выбор метода захвата объектов на изображении и оценки их позы и реализовал его.

    В процессе проектирования были сделаны схемы алгоритмов, приведено их математическое обоснование. Программное обеспечение разработана с использованием современных инструментов и технологий.

    Приведенные расчеты и программное обеспечение --- это результат высокоэффективной работы над темой и умения использовать техническую литературу и применять на практике знания, полученные за годы обучения в университете.

    Работа над проектом велась ритмично и в соответствии с календарным графиком. Пояснительная записка и графический материал оформлены аккуратно и в соответствии с требованиями стандартов.

    В заключение следует отметить, что дипломный проект соответствует техническому заданию и отличается глубокой проработкой темы и выполнен с применением современных прогрессивных технологий.

    Считаю, что Чернецкий И. Н. освоил технику разработки программного обеспечения, подготовлен к самостоятельной работе по специальности 1-31 03 04 ``Информатика'' и заслуживает присвоения квалификации математика-системного программиста.

    \vfill
    \noindent
    \begin{minipage}{0.55\textwidth}
      \begin{flushleft}
        Руководитель проекта:\\
        магистр техн. наук, ассистент\\
        кафедры информатики БГУИР
      \end{flushleft}
    \end{minipage}
    \begin{minipage}{0.4\textwidth}
      \begin{flushright}
        \underline{\hspace*{3cm}} В.В. Шендер
      \end{flushright}
    \end{minipage}
  }
\end{singlespace}


\newpage
