\begin{thebibliography}{99}

\bibitem{forsight04}
  Д. Форсайт, Ж. Понс,
  \emph{Компьютерное зрение. Современный подход}.
  М.: Издательский дом ``Вильямс'',
  2004.

\bibitem{shapiro06}
  Л. Шапиро, Дж. Стокман,
  \emph{Компьютерное зрение}.
  М.: БИНОМ. Лаборатория знаний,
  2006.

\bibitem{andriluka09}
  M. Andriluka, S. Roth, B. Schiele,
  \emph{Pictorial structures revisited: people detection and articulated pose estimation}.
  CVPR,
  2009.

\bibitem{andriluka08}
  M. Andriluka, S. Roth, B. Schiele,
  \emph{People-tracking-by-detection and people-detection-by-tracking}.
  CVPR,
  2008.

\bibitem{belongie02}
  S. Belongie, J. Malik, J. Puzicha,
  \emph{Shape matching and object recognition using shape contexts}.
  PAMI,
  2002.

\bibitem{belongie00}
  S. Belongie, J. Malik, J. Puzicha,
  \emph{Shape context: a new descriptor for shape matching and object recognition}.
  NIPS,
  2000.

\bibitem{felzenszwalb05}
  P.F. Felzenszwalb, D.P. Huttenlocher,
  \emph{Pitorial structures for object recognition}.
  IJCV,
  2005.

\bibitem{freund99}
  Y. Freund, R.E. Schapire,
  \emph{A Short Introduction to Boosting}.
  Journal of Japanese Society for Artificial Intelligence, 14(5):771--780, September,
  1999.

\bibitem{viola01}
  P. Viola, M. Jones,
  \emph{Robust Real-time Object Detection}.
  In Proc. 2nd Int'l Workshop on Statistical and Computational Theories of Vision --- Modeling, Learning, Computing and Sampling, Vancouver, Canada,
  July 2001.

\bibitem{rosset04}
  Rosset, Zhu and Hastie,
  \emph{Boosting as a Regularized Path to a Maximum Margin Classifier}.
  Journal of Machine Learning Research 5 (2004) 941-973,
  2004.

\bibitem{mihnuk07}
  Михнюк Т.Ф.,
  \emph{Охрана труда и основы экологии}.
  Минск: ``Вышэйшая школа'',
  2007.

\bibitem{devisilov09}
  Девисилов В.А.,
  \emph{Охрана труда}.
  М.: ФОРУМ,
  2009.

\bibitem{belov09}
  Белов С.В.,
  \emph{Безопасность жизнедеятельности}.
  М.: Высшая школа,
  2007.

\bibitem{decree432}
  Указ Президента РБ №432 от 31 августа 2009 года,
  \emph{О некоторых вопросах приобретения имущественных прав на результаты научно-технической деятельности и распоряжения этими правами}.
  \href{http://president.gov.by/press76885.html}{http://president.gov.by/press76885.html}.

\bibitem{palitsyn06}
  Палицын В.А.,
  \emph{Технико-экономическое обоснование дипломных проектов. В 4-х частях. Часть 4: проекты программного обеспечения}.
  Мн.: БГУИР,
  2006.

\end{thebibliography}
